\documentclass[10pt,a4page,headings,openany,%notitlepage,
oneside
%,landscape
,twocolumn
]{report}
\usepackage[utf8]{inputenc}
\usepackage[czech]{babel}
\usepackage[T1]{fontenc}
\usepackage{amsmath}
\usepackage{cmap}
\usepackage{color}
\usepackage{pdfpages}
%\usepackage{a4wide}
\usepackage{hyperref}
\usepackage{amsfonts}
\usepackage[left=1.5cm,right=1.5cm,top=2cm,bottom=2cm]{geometry}
\usepackage{amssymb} 
\usepackage{graphicx}
\pagestyle{headings}
\usepackage[pagestyles]{titlesec}
\titleformat{\chapter}[block]{\bfseries\Huge}{\thechapter}{10pt}{\vspace{-20pt}}[\vspace{-0.5ex}]
\setlength{\parindent}{0cm}
\author{Jakub Dokulil}
\title{Literatura a čeština}
\begin{document}
\pagenumbering{Roman}
\maketitle
\tableofcontents
\newpage
\pagenumbering{arabic}
\part{2.ročník}

\chapter{Realismus}

\pagestyle{headings}
\section{Ruský realismus}
\subsection{Lev Nikolajevič Tolstoj}
\textbf{\textit{Kreutzova Sonáta}}
\subsection{Anton Pavlovič Čechov}
Prozaik, dramatik
Zemřel na TBC
Tipické jsou pro něj: psychologičnost, dramatičnost u povídek, humor, pojmenování postav 
\subsubsection{Chameleon}
Tři sestry
Sestry Olga, Máša a Irina + bratr Andrej
Dílo se zabývá tématy typu smysl života, hledání lásky, touha po jiném životě. Během děje všechny postavy postupně přichází o své sny, a majetek zatímco sní o lepším životě.
\subsubsection{Smrt úředníkova (povídka)}
Stupidní povídka?
Děj je založen na přecenění toho že si žilžakin kýchnul a měl potřebu se pořád omlouvat generálovi.
Toto téma ukazuje až přehnaný zvyk se omlouvat a ponižovat za každou prkotinu
Jména popisují lidské vlastnosti, avšak nevidím zde souvislost s charakteristikou postav.
\section{Francouzský realismus}
\subsection{Gustav Flaubert}
Studoval práva v paříži\\
Prozaik\\
\\Návrat do rodného kreje s navštěvami paříže
\\Milostný vztah s o deset let starší ženou
\\Pilný, nesnadno tvořil
\\Oproti Balzacovi nezachycoval takový průřez společnosti, zachycoval pouze jednu vrstvu
\\Psychologičnost
\subsubsection{Paní Bobaryová}
Ema Bobaryová se provdává za venkovského dobře finančně zajištěného lékaře. Z manželových peněz si vytváří snový svět, také se v jejím životě objeví několik milenců. Postupně se zadlužuje, načež spáchá sebevraždu
\section{Naturalismus}
\begin{itemize}
\item Klade důraz na \textbf{biologičnost}
\item \emph{determinace} (předurčování)
\begin{itemize}
\item dědičnost \item prostředí \item doba
\end{itemize}
\item výsledky lidského jednaní 
\end{itemize}
\subsection{Émile Zola}
Francouzský prozaik a novinář představující naturalismus. Proslavil se v \emph{Dreyfusově aféře}, v té době se začíná používat pojem \textit{intelektuál} (hanlivý význam).\\
V jeho dílech \textbf{popisuje člověka v jeho prostředí}, mluví se o tom jako o lidském přírodopisu. Vydává cyklus \emph{\textbf{Les Rougon-Macquart}}, knihy jež propojují postavy.
\subsubsection{Zabiják}
\textbf{Postavy:}Gervaisa, Lauthier, Coupeau, Naua.\\
Gervaisa se provdává klempíře Coupeaua dobře si žijou, načež on spadne ze střechy, a upije se k smrti. Pak se Gervaisa vdá za Lauthiera a začnou také pít. Z jejich dcery, Nauy se stane prostitutka.
\chapter{česká literatura 2. poloviny 19. stol.}
\section{pojmy}

\textbf{Absolutismus} je vláda osoby (či strany), která nemá nijak omezenou moc.\\
\textbf{Almanach} je publikace skládající se z tématicky příbuzných částí.\\
\textbf{Říjnový diplom} je prohlášení ve kterém se r. 1861 \textsc{Fr. Jozef I} zříkává absolutismu a zavádí říšskou radu.\\
\textbf{Prosincová ústava} je soubor ústavních zákon vydaných r. 1867, přináší 
\begin{itemize}
\item posiluje pravomoci říšské rady (dělá z ní skutečný parlament)
\item zavádí svobodu vyznání
\item zajišťuje nezávislost soudů
\end{itemize}
\paragraph{Alexandr Bach} se rozhoduje pro cenzuru a omezení publicistiky, což ovlivňuje literaturu. Byl odvolán r. 1859.
\section{Májovci}
\begin{tabular}{cl}
mladší & Neruda \\ 
 & hálek \\ 
 & Světlák \\ 
\hline
starší & Němcová \\ 
 & Erben \\ 
 & Frič \\ 
\end{tabular} 
%\caption{rozdělení představitelů}
\subsection{Almanach Máj}
Vzniká r.1858.
Je soubor básní od \textsc{Jana Nerudy} a \textsc{Vítězslava Hálek}.
Jeho redaktorem byl \textsc{Jozef Barák}.
Cílem bylo vyjádřit nespokojenost s cenzurou, vytvoření něčeho nového v literatuře, přínos realistických tendencí do české literatury. Název odkazuje na \textbf{Máj} od \textsc{Karla Hynka Máchy}, jakožto revolucionáři.

\subsection{Jan Neruda}

\paragraph{knihy veršů}
Sbírka výpravných, lyrických a časových a příležitostných básní. 

\section{Generace Ruchovců a Lumírovců}
Autoři se v závislosti na kosmopolitní Lumírovce (vydávající časopisy) a nacionalistické Ruchovce (vydávali almanachy).

\subsection{Ruchovci}

\subsubsection{Svatopluk Čech}
Byl český básník, prozail a novinář. Poezii psal jak lyrickou, tak lyrickoepickou a také alegorickou.

\paragraph{Pravý výlet pana Broučka do Měsíce}
\paragraph{Nový epochální výlet pana broučka, tentokrát do 15. století}
Je prozaické dílo, odehrávající se v Praze. V díle vystupují postavy \textsc{Matěje Broučka} a \textsc{Domšíka} neboli (Janka od zvonu). 

Pan Brouček je tipický měšťan žijící v 19. století, je člověkem který se chce mít dobře, je bezzásadový, bez víry a vlastenectví. Jendou to v hospodě přežene a najednou se probudí v 15. století. 

\section{Kritický realismus}
Předchází mu Němcová, Světlák, Neruda. Tento směr se částěčně překrývá s \textbf{modernou}(impresionismus).


Odlišnoti od světa:
\begin{description}


\item[není komplexní] nezachycuje všechny spol. vrstvy, zachycuje pouze výsek společnosti.
\item[Využití kronik, záznamů a novin]
\item\textbf{zabývá se určitou historickou a geografickou situaci}
\item[Využívá archaismů a dialektu]
\end{description}

Kritický realismus má 3 hlavní proudy:
\begin{itemize}
\item venkovská próza
	\begin{itemize}
	\item \textsc{K.V. Rais} 
	\item \textsc{Tereza Nováková} (Litomyšl, Polička, zaměření se na mužské postavy)
	\item \textsc{J. Š. Buar} (chodsko)
	\end{itemize}
\item městská próza
\begin{itemize}
\item \textsc{Z. Winter}
\end{itemize}
\item historická \begin{itemize}
	\item Husitsví
	\item humanismus
	\item Bílá hora, pobělohorská doba
	\item Národní obrození
	\end{itemize}

\end{itemize}

\subsection{Venkovská Próza}
\paragraph{\textsc{K. V. Rais}} Autor venkovské prózy často píše o oblasti Podkrkonoší, a Hlinecka, Autor souboru \textbf{Výminkáři}

\subparagraph{Výminkáři} Soubor se zabývá mezigeneračními vztahy a jejich problémy a majetkovými vztahy.

\subsection{Historická próza}
\paragraph{Zikmund Winter} Životopis \emph{Mistr Kampanus}

\subparagraph{Mistr Kampanus} životopis \textsc{Jana Kampana Vodňanského},  rektora UK. Dílo je psáno velmi zastaralý jazykem.

\paragraph{A. Jirásek} Zabýval se mýty a pověstmi (\textbf{staré pověsti české}), husitstvím (proti všem), rekatolizace a doba pobělohorská(Temno, psohlavci), národním obrození (I.L. Věk).
\subparagraph{Temno} 
 
 
\section[Národní divadlo]{Generace Národního divadla}
\subsection{\textsc{Ladislav Stroupežnický}}
\paragraph{Naši furianti} komediální drama, kde se starosta s 1. radním kteří se jednak předhání v tom kdo jejich dětem dá větší věno a také spor o to kdo se stane ponocným.

\subsection{\textsc{A. }a\textsc{ V. Mrštíkové}}
Představitelé Realismu, drama Maryša.

%%%%%%%% T Ř E Ť Á K %%%%%%%%%%%%%%%%%%%

\part{3. ročník}
\chapter{Moderna}
\section{Spojené státy}
\subsection[E. A. Poe]{Edgar Alan Poe}

V mládí byl v pěstounské péči, studoval v USA a VB. Později pracoval jako redaktor avšak žil v chudobě, v důsledku toho umírá r. 1847. Představoval americký romantizmus

\paragraph{Povídka Černý kocour} Popisuje jak vypravěč si pořizuje kocoura a následně se stává alkoholikem, v důsledku toho zmrzačí a později kocoura. Později získá nového kocoura, skoro stejného jako byl ten první, při pokusu ho zabít zabíjí svou ženu a kocoura zahání. Nakonec se mu podařilo vraždu utajit.
%%%%%%%%%%%%%%%%%%%%%%%%%%%%%%%%%%%%%%%%%%%%%%%%%%%%%%%%%%%%%%%%%%%%
%%%%%%%%%%% N O V Á   Č Á S T %%%%%%%%%%%%%%%%%%%%%%%%%%%%%%%%%%%%%%%%%%%%%%%%%
%%%%%%%%%%%%%%%%%%%%%%%%%%%%%%%%%%%%%%%%%%%%%%%%%%%%%%%%%%%%%%%%%%%%

\part{4. ročník\\maturitaaaaaaaaaaaaaaaaaaaaaaa!!!!!!!!!!!!!!!!!!}

\chapter{česká meziválečná literatura}

\section{směry}

\subsection{Poetizmus}
Původně český avantgardní umělecký směr, též způsob života. Texty jsou bez interpunkce a je použita metoda asociací. Cílem je změnit společnost v harmonickou jednotu. Motivy z oblasti lidové zábavy(kino, cirkus, poutě). 

\subsection{Surrealismus}
R. 1934 zakládá V. Nvzal Surrealistickou skupinu ČSR.

\section{Jiří Wolker}
sbírka host do domu, básně, ovlivněn Apolinérem, sbírka těžká hodina\footnote{balada o nenarozeném dítěti}

\section{Jozef Hora}
Básník, dramatik prozaik. Pocházel z chudé rolnické rodiny. Studoval práva, nedokončil je.

\textsf{Proletářské období}
\begin{itemize}
\item dělníci, revoluce\dots
\item bouřlivé jaro
\end{itemize}

\textsf{Poetizmus}
\begin{itemize}
\item sbírka Itálie (hodně emocí\dots) 
\item odklon od proletářského umění

\end{itemize}

\textsf{Poezie času, smrti a ticha}
\begin{itemize}
\item odklon od všeho co do té doby dělal
\item Jan houslista -- básnická skladba (ukazuje ohrožení českého národa nacizmem)
\end{itemize}

\section{Vítězslav Nezval}
Hlavní představitel avantgardní poezie, zároveň prozaik a dramatik. Člen Devětsilu. Ke konci života píše Socialistický realismus\footnote{Ofiko směr komunistické diktatury.}.


\textsf{Poetizmus}
\begin{itemize}
\item Manifest poetismu \emph{Papoušk na motocyklu}
\item podivuhodný kouzelník
\item Sbírka Básně Noci (Edison -- oslava vědce)
\end{itemize}

\textsf{Surrealismus}
\begin{itemize}
\item Přichází s ekon krizí.
\item Absolutní hrobař, žena v mn. čísle (navazuje na S. Dalího\dots)
\end{itemize}


\textsf{Válečná léta}
\begin{itemize}
\item Protinacistická básnická sbírka \emph{5 minut za městem}
\item Manon Lescaut -- div. hra
\end{itemize}

\section{J. Seifert}

Básník, získal Nobelovu cenu za Literaturu (1984). Poetizmus -- surrealizmus -- intimní lyrika

Narodil se v Praze, pocházel z chudé dělnické rodiny. Vstoupil do komunistické strany a publikoval v Rudém právu. R. 1929 byl vyloučen z komunistické strany.

\textsf{Prol. poezie}
\begin{itemize}
\item sbírky město v slzách, samá láska -- věří v spravedlivé uspořádání společnosti
\end{itemize}
\textsf{Poetizmus}
\begin{itemize}
\item sbírky Na vlnách TSF -- metoda asociací a důraz na obrazovost
\item slavík zpívá špatně -- odklon od poetizmu
\end{itemize}	
\vspace{2cm}
\textsf{Intimní lyrika} -- 30. léta
\begin{itemize}
\item originální směr
\item Jaro, sbohem;  Jablko z klína; Ruce venušiny
\item Erotická poezie, melodické verše, které se podobají lid písním 
\item psáno v rýmech, pravidelné verše

\end{itemize}

\textsf{Nacistická okupace}
\begin{itemize}
\item Zhasněte světla, Vějíř Boženy Němcové
\end{itemize}

\textsf{Fáze od 50. letech}
\begin{itemize}
\item vrací se ke způsobu psaní z 30. let, píše melodické srozumitelné verše -- Maminka
\item Píseň o Viktorce -- o životě postavy z románu Babička\footnote{Byla komunisty zakázána.}
\item Morový sloup -- reakce na 21. srpen 1968
\item Všecky krásy světa -- vzpomínková próza
\end{itemize}

\section{Konstantnin bíbl}
Účastnil se WW1, po válce se začal věnoval poetizmu, na konci 20. let ho opustil, začal psát Surrealistická díla.

\begin{itemize}
\item \emph{S lodí, jež dováží čaj a kávu} -- exotické motivy
\item \emph{Nový ikaros} -- je řazena k surrealizmu, tragická pochmurná skladba
\end{itemize}

\section{Spiritualistická poezie}
\subsection{F. Halas}
Spojen s Kunštátskem. Byl komunista.

Kohout plaší smrt -- záměrně píše nesrozumitelné věty.

Staré ženy -- skladba, ve které soucítí s procesem stárnutí u žen.

Během okupace píše sbírku naše paní božena němcová

Já se tam vrátím -- básnická póza, píše o krásách kunštátska

\subsection{Vladimír Holan}
Básník a překladatel. Psal intelektuálně náročnou poezii. používá velklé množství oxymorónů a paradoxů

Kamenní přicházíš\dots, první testament a Terezka planetová.

\section{Křesťansky píšíci autoři}

\subsection{Jan Zahradníček}
Psal křesťansky zaměřenou tvorbu. Byl vězněn komunisty.
La Saletta, Znamení moci, Čtyři léta, Dům strach, čtyři léta



\subsection{Jakub Deml}
Básník a rozaik, který vykonával kněžské povolání.

\chapter{Česká meziválečná próza}
\section{Demokratický proud}
Autoři, kteří jsou součástí tohoto směru, hájili demokratické hodnoty během existence první republiky.

\subsection{Karel Čapek}
Prozaik, dramatik (dramaturg Divadlo na Vinohradech) a novinář(Lidové noviny). Umírá před vánoci 1938. 

Témata: \textsf{varování před zneužitím vědy a techniky, upozorňování na hrozbu fašizmu}
\paragraph{R.U.R.}(Rossumovi univerzální roboti) antiutopické drama s prvky sci-fi, čapek varuje, že techn. vynálezy se mohou obrátit proti lidstvu (Robot -- odvozeno od slova \emph{robota}), roboti měli za lidi vykonávat práci a válčit. ROboti vvraždí lidstvo a vzbouří se až na jednoho starého vynálezce Alquista vyvraždí lidstvo. Kniha končí tím, že se do sebe dva roboti zamilují a stanou se lidmi.

\paragraph{Ze života hmyzu} napsal ji karel čapek ve spolupráci s bratrem jozefem. Jsou zde využity alegorie.

\paragraph{Krakatit} jistý doktor vynalezne třeskavinu, zvanou krakatit. Tématem je odpovědnost vědců za své vynálezy.

\paragraph{Povídky z jedné kapsy, Povídky z druhé kapsy} Kriminální povídky, téma poznatelnost pravdy.

\paragraph{Hordubal} Román o borcovi se se vrátí z ameriky pak  je zabit a žena ho podváděla.

\paragraph{Továrna na absolutno} antiutopický román, je sestrojen nadkritický reaktor, který vyzařuje temnou energii a vede lidstvo do záhuby

\paragraph{Věc Makropulos} Utopické drama. 
Čapek vyjadřuje svůj názor, že lid. život má smysl pouze pokud je konečný.

\paragraph{Válka s mloky} protiválečný román, ve kterém Čapek reaguje na A. Hitlera je protinacistický. Mloci se rozlézají po 

\paragraph{Bílá nemoc} je nepříjemné onemocnění, které zasahuje lidi. Dr. Galén zná lék, ten ho ovšem nechce vydat pouze pokud skončí. protivík maršál je ochoten přistoupit na Galénovy podmínky, avšak, když nese lék Maršálovi tak je ušlapán. Protiválečná divadelní hra.

\textsf{Doktor Galén}--pacifista, \textsf{Maršál}--militarista

\paragraph{Působení v LN} Vytvořil nový slohový útvar--sloupek\footnote{Je to krátká reakce na aktuální problematiku s Ironii a sarkazmem}. 

\subsection{Jozef Čapek}

Kubistický malíř a spisovatel, také přispíval do LN a bojoval proti nacizmu. 1. září 1939 byl umístěn do koncentračního tábora. Umírá r. 1945 na tifus.

\paragraph{Stín kapradiny} je baladická novela

\paragraph{Kulhavý poutník}--filozofické eseje

Básně z konc. tábora.

\subsection{Karel Poláček}
Porzaik novinář. Působil v LN. Byl žid. Píše o životě na maloměstě.

\paragraph{Bylo nás pět.} Rozsáhlá próza z maloměsta. Hlavní postavou a zároveň vypravěčem je chlapec Petr Bajza. Komičnost je založena na jazyce, který používají postavy (kombinace archaické spisovné češtiny s nespisovnou češtinou) a používání mluvy dospělých dětmi. Maloměsto popsáno očima dítěte.

\subsection{Eduard Bass}

Prozaik a novinář, šéfredaktor LN. Vytvořil rozhlásku (ve verších napsaná reakce na uplynulý týden).

\paragraph{Cirkus Humberto:} historický román, který se odehrává v průběhu 19. století. Hl postavou je Vašek karas, který se vypracuje z pomocníka až na ředitele v cirkuse.

\subsection{Vladislav Vančura}

\paragraph{Rozmarné léto} Hojný výskyt přechodníků, autor vypráví děj nespojitě. Důra je své manželce nevěrný, 

\paragraph{Markéta Lazarová} Baladický román a imaginativní próza. Odehrává se ve středověku na mladoboleslavsku. Hlavní postavou je Markéta z rodu Lazarů (dcera loupeivého rytíře) jejím přáním je stát se jeptiškou. Markétu unese Mikoláš Kozlík, znásilní ji a ona se do něj zamiluje.	Mikoláš je porpaven a Markéta očekv jeho dítě. Erformový vypravěč, který zasahuje do děje. Složitá mluva s archaismy a metaforami.

\subsection{Ivan Olbracht}
Levicově zaměřený autor. Vyloučen z KSČ. 

\paragraph{Anna Proletářka} komunistický text.

\paragraph{Nikola Šuhaj loupežník} Baladický román


\section{Psych. próza}

\subsection{Jaroslav Havlíček}
Autor psych. prózy se znaky naturalizmu. Přejímá od naturalistů myšlenku, že člověk je podmíněn společností a původem. Prostředí maloměsta (kritika poměrů). 
\paragraph{Petroleové lampy} Psyhcologický román, kde hlavní postava, Štěpka, je svérázná mladá žena, nepřijímána v jejím okolí. Štěpka touží po sňatku a dítěti, proto si vezme svého bratrance Pavla, který má sifilis, a proto se dítěte nedočká.

\paragraph{Neviditelný} Psychologický román s hororovou atmosférou. Tématem je dědičnost psych poruch. 

\paragraph{Helimadoe} Starý muž vzpomíná na dospívání, které prožil v blízkosti 5 sester

\subsection{Glazarová}

\paragraph{Advent} Františka, svobodná matka, hledá svého syna. Františka se vdá za sedláka, který týrá jejího syna.

\subsection{Egon Hostovský}
Žid. původ, musel dvakrát emigrovat. Tématika odcizení a emigrace. Popisuje situaci muže, který odchází do zahraničí.

\paragraph{Žhář} pojednává o nemoci pyromanie,


\paragraph{Případ prof. Körnera} Psych román, vystupuje v něm mlaý, přecitlivělý, introvertní a labilní.

\section{Katolická próza} Psalio víře a křsťanství

\subsection{Jaroslav Durych} Konzervativní katolický prozaik. Simpatizoval s španělskými fašismy. Odmítá politiku první republiky a kritizuje TGM a K. Čapka.

\paragraph{Rekviem} Barokní soubor povídek, po smrti albrechta z valdštějna.

\subsection{Jan Čep}

\paragraph{Druhý domov}Hlavní postavou děti, nadšené životem, či staří lidé, kteří se s životem loučí.

\section{Expresionistická próza}

\subsection{Richard Weiner}

Tématem je dvojnictví.	

\paragraph{Lítice} Lítice -- bohyně výčitek

\section{České meziválečné drama}
Po 1. světové válce začali v divadlech působit čeští dramatici. 

\textsc{Karel Hugo Hilar} -- ředinat ND

1926 -- divadelní sekce Devětsilu zakládá osvobozenecké divadlo. Avantgardní divadlo (poetizmus, dadaismus, futurismus). R. 1927 přichází do divadla \textsc{Jiří Voskovec} a \textsc{Jan Werich}. Nejslavnější hra Vest pocket Revue. Obsahovaly bezpředmětnou komiku. 1931 se přidává jazzový skladatel \textsc{Jaroslav Ježek}(život je jen náhoda, \dots) Později psali polit. satiru. Součástí byla \emph{forbína} -- předscéna (improvizovaný dialog na aktuální téma, komunikace s publikem)

Hra Caesar -- polit satira v níž kritizují diktátory (zde konkr Benita mussoliniho).

1938 divadlo zavřeno, autoři emigrují do ameriky.

Roku  1933 bylo založeno divadlo \textbf{D34}. Založeno \textsc{Emilem Františkem Burianem}. Byla to avantgardní divadelní scéna. Dramatizoval básnické a prozaické texty. Např. Máj, Krysař... Manon Lescaut -- vysoká návštěvnost. Byla obdivována krása češtiny - forma protestu.

\chapter{Světová próza 2. poloviny 20. století}

Po skončení 2. sv války příchází studená válka, ta se také projevuje v umění. Svět se dělí na západní demokratický blok a východní blok ovládaný SSSR, propagován byl \emph{socialistický realizmus}. Ve východním bloku se literatura dělí do tří proudů:

\begin{description}
\item[Domácí literatura] -- díla, která prošla cenzurou
\item[Samizdatová literatura] - ilegální lit.
\item[Exilová literatura] -- také ilegální, většinou těch, kteří emigrovali, či texty z domova byly vydány v zahraničí

\end{description}

\section[Druhá světová válka v próze]{Obraz druhé světové války ve světové próze}

Téma: \textsf{druhá světová válka}

\paragraph{Deník Anne Frankové} deníkové záznamy židovskék dívky (1929--2925). Když byli malí utekli do nizozemí.

1) téma, motivy vypravěče -- \emph{Vyplňování volného času, jak se zabavit }

2)téma, motivy, charakteristika anne -- \emph{vztahy v ovropě, odlišnosti, }

\paragraph{Wladyslslaw Szpilman:} Klavírista polského původu.

\paragraph{Sophiina volba}Vypravěč Stingo se seznámí v polkou Sopfií, která má za přítele žida Nathana. Sofie přežila osvětim a Březinku kde byla i se svými dětmi. V osvětimi se střetává se sadistickým lékařem, který ji donutí zvilit, které z dětí zemře. Žádné z dětí nakonec nepřežilo. Celý život se trápí pocitem viny a se svým přítelem mají spoustu traumat z konc tábora nakonec oba spolu spáchají sebevraždu.

\paragraph{Jozeph Heller}

\subparagraph{Hlava XXII}

\paragraph{Norman Mailer} Americký letec. \textsf{Nazí a mrtví} -- ve válce nezáleží na jedinci.

\paragraph{Kurt Vonnegut} Jatka číslo 5 -- román o bombardování Drážďan.

\section{Existenconalismus}

\subsection{Jean-Paul Sartre}

\paragraph{Bytí a nicota} -- traktát (pojednání)
\vspace{12pt}

\textbf{Mouchy, Za zavřenými dveřmi}

\subsection{Albert Camus}Francouzský spisovatel a filozof původem z Aelžírska. Získal a přijal Nobelovu cenu za literaturu.

\paragraph{Mýtus o Sisyfovi} Je filozofická esej, čerpá z Antické mitologie. Sisyfos každý den valý kámen do kopce aby se zkutálel zpět domů. Podle Camuho je život sisyfofská práce. Odmítá jako řešení situace sebevraždu i víru v boha.

\paragraph{Cizinec} Román. Hlavní postavou a zároveň vypravěčem je úředník \textsc{Mersaud}. Ten se řídí pouze svými instinkty a pudy. Nedbal na své emoce a řídil se pouze svými instinkty. Byl odsouzen k trestu smrti, protože vůbec nedodržoval konvence.

\part{Jazyk}

\begin{enumerate}
\item gramatika
\begin{enumerate}
\item Hláskosloví (fonetika, fonologie--vliv hlásek na význam, ortofonie--artikulace, ortoepie--spisovná výslovnost, ortografie--pravopis)
\item morfologie -- tvarosloví
\item syntax
\end{enumerate}
\item Lexikologie
\begin{enumerate}
\item Sémantika--význam slov
\item slovotvorba
\item lexikografie--slovníky
\item frazeologie
\end{enumerate}
\item Stylistika
\begin{enumerate}
\item Funkční styl (prostěsdělovací, řečnický, odborný, umělecký, publicistický, administrativní) 
\item slohové postupy (informační, popisný, vyprávěcí, úvahový, výkladový)
\item sloh. útvar
\end{enumerate}
\item Historická mluvnice -- nauka o vývoji jazyka
\item Dialektologie
\end{enumerate}

	\chapter{Modalita}
	Postojová
	\begin{itemize}
	\item oznamovací \item tázací \item přací \item rozkazovací
	\end{itemize}
	Modalitu lze dále hodnotit také podle těchto kritérií
	\begin{itemize}
	\item jistotní (vyjádření jistoty) \item preferenční (vyjádření mé vlastní vůle) \item \textit{evaluativní} (hodnotící, vyjádření hodnotícího postoje)
	\end{itemize}


\chapter{Odchylky od větné stavby}

\begin{description}
\item[elipsa]
\item[apozioteze]
\item[osamostatňování části výpovědi]
\item[vsuvka]
\end{description}

\paragraph{oprava vět}

\emph{Gratuluji vám za vaše úspěchy.} --  Gratuluji vám \textbf{k}  vašim úspěchům\\
\emph{Připadla na mě milá příležitost \dots} -- Připadla \textbf{mi} milá příležitost \dots\\
\emph{Na návštěvě byli všichni mimo Jendy} -- \dots všichni \textbf{krom}  Jendy.
\vspace{0.5cm}

Když naši hráči, zvyklí na kvalitní hřiště, přijdou do ciziny, špatně se jim hraje. Často když lidé dělají nějakou prác poprvé, tak se jim nedaří dobře. 

\chapter{Odborný text}
Sdělovat inf. z různých oblastí lidského poznánní a co nejpřesněji vzdělávat.

Podstyly:
\begin{description}
\item[Vědecký]Je velmi odborný, rozumí jim většinou jen odborníci.
\item[Prakticky odborný]Je srozumitelný pro většinu veřejnosti.
\item[Popularizační] Ten cílí pro šíření informací pro široké publikum.
\end{description}

\section{Práce s informacemi}

\begin{description}
\item[Osnova]
\item[Výpisek] doslova vypsaná část textu
\item[Výtah] přeparafrázovaná (námi přepsaná) část textu
\end{description}

\subsection{Parafráze textu}

Příklad z článku \cite{esej}

\begin{quotation}

František Daneš ve svém článku pro časopis vesmír\cite{esej} s názve \emph{\uv{Jak je tomu s esejem?}}, nepojednává o tom že slovo esej kolísá mezi ženským a mužským rodem, tak jak to jména končící měkkou souhláskou dělávají, nýbrž se zabývá významem tohoto slova.

Toto slovo má úplný původ v latině (\textit{exagnium} -- zvažovat) odkud ho přejali Francouzi a posléze Angličani. V Česku však nemá příliš tradici. Jeho význam je často chápán jako útvar který je jak umělecký tak odborný, navíc právě takový V našem jazyce pak nejbližším významem tomuto slovu je jméno úvaha. Ne však v Anglii kde \uv{write an essay} znamená psát studentskou práci na dané téma, a takovéto práce se v Anglii píší mnohem častěji naž u nás.

Prvně toto slovo použil francouz Michela de Mintaigne později toto slovo rozmohlo a \dots 
\end{quotation}

\begin{thebibliography}{9}

\bibitem{esej} DANEŠ, František. Jak je tomu s Esejem? \textit{Vesmír}. 1994,\textbf{73} (12).

\end{thebibliography}

\section{Jazyková norma a Kodifikace}

\paragraph{Jazyková norma} jsou jazykové prostředky, které mluvčí považuje za spisovné. (To co si každýmyslí že je spysovné.) 

Jazykový úzus -- návyk na špatné či chybné používání jazykových prostředků.

\paragraph{Kodifikace} soubor pravidel pro používání jazyka v oficiálích jazykových přírůčkách. (Pravidla českého pravopisu)

\paragraph{Jazyková kultura} -- péče a spisovný jazyk, kultura jazykového vyjadřování	

\end{document}

